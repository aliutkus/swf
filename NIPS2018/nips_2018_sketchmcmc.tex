\documentclass{article}

% if you need to pass options to natbib, use, e.g.:
% \PassOptionsToPackage{numbers, compress}{natbib}
% before loading nips_2018

% ready for submission
\usepackage{nips_2018}

% to compile a preprint version, e.g., for submission to arXiv, add
% add the [preprint] option:
% \usepackage[preprint]{nips_2018}

% to compile a camera-ready version, add the [final] option, e.g.:
% \usepackage[final]{nips_2018}

% to avoid loading the natbib package, add option nonatbib:
% \usepackage[nonatbib]{nips_2018}

\usepackage[utf8]{inputenc} % allow utf-8 input
\usepackage[T1]{fontenc}    % use 8-bit T1 fonts
\usepackage{hyperref}       % hyperlinks
\usepackage{url}            % simple URL typesetting
\usepackage{booktabs}       % professional-quality tables
\usepackage{amsfonts}       % blackboard math symbols
\usepackage{nicefrac}       % compact symbols for 1/2, etc.
\usepackage{microtype}      % microtypography
\usepackage{amsmath}
\usepackage{amssymb}

\newcommand{\W}{{\cal W}_2}
\newcommand{\F}{{\cal F}}
\newcommand{\He}{{\cal H}}
\newcommand{\SW}{{\cal S}{\cal W}_2}

\DeclareMathOperator*{\argmin}{arg\min}



\title{Sliced-Wasserstein Flows: Learning Generative Models via Single Data Pass with Guarantees}

% The \author macro works with any number of authors. There are two
% commands used to separate the names and addresses of multiple
% authors: \And and \AND.
%
% Using \And between authors leaves it to LaTeX to determine where to
% break the lines. Using \AND forces a line break at that point. So,
% if LaTeX puts 3 of 4 authors names on the first line, and the last
% on the second line, try using \AND instead of \And before the third
% author name.

\author{
  David S.~Hippocampus\thanks{Use footnote for providing further
    information about author (webpage, alternative
    address)---\emph{not} for acknowledging funding agencies.} \\
  Department of Computer Science\\
  Cranberry-Lemon University\\
  Pittsburgh, PA 15213 \\
  \texttt{hippo@cs.cranberry-lemon.edu} \\
  %% examples of more authors
  %% \And
  %% Coauthor \\
  %% Affiliation \\
  %% Address \\
  %% \texttt{email} \\
  %% \AND
  %% Coauthor \\
  %% Affiliation \\
  %% Address \\
  %% \texttt{email} \\
  %% \And
  %% Coauthor \\
  %% Affiliation \\
  %% Address \\
  %% \texttt{email} \\
  %% \And
  %% Coauthor \\
  %% Affiliation \\
  %% Address \\
  %% \texttt{email} \\
}

\begin{document}
% \nipsfinalcopy is no longer used

\maketitle

\begin{abstract}

\end{abstract}

\section{Introduction}

\begin{itemize}
\item Short intro to implicit generative models
\item Connection with optimal transport \cite{genevay2017gan}, several other papers
\item Information about sliced-Wasserstein distance and usage in generative modeling \cite{bonnotte2013unidimensional,kolouri2018sliced,wu2017generative}
\item Contributions of the current paper
\begin{itemize}
\item Development of a novel gradient flow for generative modeling purpose
\item Establish the connections with SDEs
\item Develop a practical way to simulate the SDE
\item Establish theoretical guarantees
\item Experimental validation
\end{itemize}
\end{itemize}

\section{Technical Background}

\begin{itemize}
\item Brief intro to gradient flows in $\W$ 
\item Fokker-Planck equations and relations to SDEs
\item Langevin equation as a special case
\item SGLD for optimization \cite{raginsky17a,zhang17b}
\end{itemize}

\section{Sliced-Wasserstein Flows for Generative Modeling}

\begin{itemize}
\item Start from the flow given in \cite{bonnotte2013unidimensional} and discuss the limitations 
\begin{itemize}
\item No convergence guarantee
\item Over-fitting risk when the target is a collections of Dirac masses
\end{itemize}
\item Motivation for the entropy-regularization (discuss the differences with the Sinkhorn distances \cite{genevay2018learning})
\item Development of the new gradient flow -- first theoretical result
\item Establish the distance between the stationary measures of the two gradient flows (second theoretical result)
\item View the new flow as a non-linear Fokker-Planck equation -- establish the connection with the appropriate SDE
\item Develop the numerical scheme 
\item Analyze the TV distance between the measure at iteration $k$ and the invariant measure
\item We don't need to see the data, we just need the projections. In addition to its computational implications, this can be crucial for privacy preserving applications
\end{itemize}

\section{Experiments}

\begin{itemize}
\item Experiments on GMM
\begin{itemize}
\item We can focus on the 2D problem
\item Monitor the likelihood per iteration
\item We need to show the behavior and the additional contribution of the entropy -- repeat experiments for different level of the Gaussian noise
\end{itemize}
\item Experiments on real data
\begin{itemize}
\item The more the number of real datasets, the better
\item Experiments on emnist and Fashion mnist
\item We should illustrate the behavior of the level of the Gaussian noise
\end{itemize}
\end{itemize}

\section{Conclusion}



\bibliography{./references.bib}
\bibliographystyle{unsrt}



\end{document}
