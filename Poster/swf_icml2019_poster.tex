\documentclass[landscape,final,a0paper,fontscale=0.265]{baposter} %fontscale=0.277


\usepackage{calc}
%\usepackage{fourier}
\newcommand\Warning{%
 \makebox[1.4em][c]{%
 \makebox[0pt][c]{\raisebox{.1em}{\small!}}%
 \makebox[0pt][c]{\color{red}\Large$\bigtriangleup$}}}
\usepackage{graphicx}
\usepackage{amsmath}
\usepackage{amssymb}
\usepackage{relsize}
\usepackage{multirow}
\usepackage{rotating}
\usepackage{bm}
\usepackage{url}
\usepackage{cancel}

\usepackage{graphicx}
\usepackage{xargs}
\usepackage{multicol}


\usepackage{enumitem}

%\usepackage{times}
%\usepackage{helvet}
%\usepackage{bookman}
\usepackage{palatino}



\usepackage{ragged2e}
\usepackage{setspace}

\usepackage{anyfontsize}


\newcommand{\mybox}[2]{%
         \begin{center}%
            \begin{tikzpicture}%
                \node[rectangle, draw=#2, top color=#2!10, bottom color=#2!10, rounded corners=5pt, inner xsep=5pt, inner ysep=5pt, outer ysep=0pt]{
                \begin{minipage}{0.97\linewidth}#1\end{minipage}};%
            \end{tikzpicture}%
         \end{center}%
}


\newcommand{\captionfont}{\footnotesize}

\newtheorem{lemma}{Lemma}
\newtheorem{theorem}{Theorem}
\newtheorem{corollary}{Corollary}


\newenvironment{assumption}{\textbf{Assumption:} }{}
\newenvironment{remark}{\textbf{Remark:} }{}


\graphicspath{{images/}{../images/}}
\usetikzlibrary{calc}

\newcommand{\SET}[1]  {\ensuremath{\mathcal{#1}}}
\newcommand{\MAT}[1]  {\ensuremath{\boldsymbol{#1}}}
\newcommand{\VEC}[1]  {\ensuremath{\boldsymbol{#1}}}
\newcommand{\Video}{\SET{V}}
\newcommand{\video}{\VEC{f}}
\newcommand{\track}{x}
\newcommand{\Track}{\SET T}
\newcommand{\LMs}{\SET L}
\newcommand{\lm}{l}
\newcommand{\PosE}{\SET P}
\newcommand{\posE}{\VEC p}
\newcommand{\negE}{\VEC n}
\newcommand{\NegE}{\SET N}
\newcommand{\Occluded}{\SET O}
\newcommand{\occluded}{o}
\usepackage{natbib}
\def\bibfont{\tiny}
\setlength{\bibsep}{0pt plus 0.3ex}

\usepackage{empheq}
\usepackage{dsfont}
\usepackage{enumitem}
\usepackage{ stmaryrd }

% \newtheorem{thm}{Theorem}

% \DeclareMathOperator*{\argmin}{arg\min}
% \DeclareMathOperator*{\argmax}{arg\max}

%%%%%%%%%%%%%%%%%%%%%%%%%%%%%%%%%%%%%%%%%%%%%%%%%%%%%%%%%%%%%%%%%%%%%%%%%%%%%%%%
%%%% Some math symbols used in the text
%%%%%%%%%%%%%%%%%%%%%%%%%%%%%%%%%%%%%%%%%%%%%%%%%%%%%%%%%%%%%%%%%%%%%%%%%%%%%%%%

%%%%%%%%%%%%%%%%%%%%%%%%%%%%%%%%%%%%%%%%%%%%%%%%%%%%%%%%%%%%%%%%%%%%%%%%%%%%%%%%
% Multicol Settings
%%%%%%%%%%%%%%%%%%%%%%%%%%%%%%%%%%%%%%%%%%%%%%%%%%%%%%%%%%%%%%%%%%%%%%%%%%%%%%%%
\setlength{\columnsep}{1.5em}
\setlength{\columnseprule}{0mm}

%%%%%%%%%%%%%%%%%%%%%%%%%%%%%%%%%%%%%%%%%%%%%%%%%%%%%%%%%%%%%%%%%%%%%%%%%%%%%%%%
% Save space in lists. Use this after the opening of the list
%%%%%%%%%%%%%%%%%%%%%%%%%%%%%%%%%%%%%%%%%%%%%%%%%%%%%%%%%%%%%%%%%%%%%%%%%%%%%%%%
\newcommand{\compresslist}{%
\setlength{\itemsep}{1pt}%
\setlength{\parskip}{0pt}%
\setlength{\parsep}{0pt}%
}

%%%%%%%%%%%%%%%%%%%%%%%%%%%%%%%%%%%%%%%%%%%%%%%%%%%%%%%%%%%%%%%%%%%%%%%%%%%%%%
%%% Begin of Document
%%%%%%%%%%%%%%%%%%%%%%%%%%%%%%%%%%%%%%%%%%%%%%%%%%%%%%%%%%%%%%%%%%%%%%%%%%%%%%


% \usepackage{xargs}

\newcommand{\supp}{Appendix}
% \newcommand{\supp}{the supplementary document}

\newcommand{\x}{\mathbf{x}}
\newcommand{\thb}{\mathbf{x}} 
\newcommand{\Ths}{{\cal X}} 
% \newcommand{\thb}{\boldsymbol{\theta}} 
% \newcommand{\Ths}{{\cal M}} 
\newcommand{\xe}{\tilde{\mathbf{x}}} 

\newcommand{\B}{\mathcal{B}}
\newcommand{\N}{\mathcal{N}}
\newcommand{\q}{\mathbf{q}}
\newcommand{\qc}{\mathbf{\bar{q}}}
\newcommand{\p}{\mathbf{p}}
\newcommand{\tb}{\mathbf{t}} 
\newcommand{\Ds}{{\cal D}} 

\newcommand{\M}{\mathbf{M}}
\newcommand{\stl}{l}
\newcommand{\Lo}{{\cal L}}
\newcommand{\Oc}{{\cal O}}
\newcommand{\Lot}{\tilde{\Lo}}
\newcommand{\y}{\mathbf{y}}
\newcommand{\Y}{{ Y}}
\newcommand{\D}{ {\cal D} }
\newcommand{\ba}[1]{b(#1,\alpha)}
\newcommand{\bta}[1]{\tilde{b}_{h,K}(#1,\alpha)}
\newcommand{\bha}[1]{\hat{b}(#1,\alpha)}
\newcommand{\rmd}{r}
\newcommand{\Sp}{\mathbb{S}}
\newcommand{\R}{\mathbb{R}}
\newcommand{\E}{\mathbb{E}}
\newcommand{\Pro}{\mathbb{P}}
\newcommand{\Pb}{\mathbf{P}}
\newcommand{\overbar}[1]{\mkern 1.5mu\overline{\mkern-1.5mu#1\mkern-1.5mu}\mkern 1.5mu}

\newcommand{\W}{{\cal W}_2}
\newcommand{\WS}{\mathbb{W}_2}
\newcommand{\F}{{\cal F}}
\newcommand{\PS}{{\cal P}}
\newcommand{\He}{{\cal H}}
\newcommand{\SW}{{\cal S}{\cal W}_2}
\newcommand{\TV}{\textnormal{TV}}
\newcommand{\KL}{\textnormal{KL}}
\newcommand{\muh}{\hat{\mu}}
\newcommand{\mub}{\bar{\mu}}

\newtheorem{thm}{Theorem}
\newtheorem{remark}{Remark}
\newtheorem{cor}{Corollary}
\newtheorem{lemma}{Lemma}
\newtheorem{prop}{Proposition}

\DeclareMathOperator*{\argmin}{arg\min}
\DeclareMathOperator*{\argmax}{arg\max}

\DeclareMathOperator{\cB}{\overline{B}}
\newtheorem{definition}{Definition}

\newcommand\simiid{\stackrel{\mathclap{\normalfont\mbox{\tiny i.i.d.}}}{\sim}}


\newcommand{\tmpeqno}{{\color{red} (TEMPEQ) }}

\newcommand{\umut}[1]{{\color{red} (#1)} }
\newcommand{\alain}[1]{{\color{blue} (#1)} }

\DeclareMathOperator{\sign}{sign}
\newcommand{\sas}{{\cal S} \alpha {\cal S} }

% \newcommand{\pb}{\bar{{\cal P}}}
% \newcommand{\pt}{\tilde{{\cal P}}}

\newcommand{\pb}{{\cal P}^\x}
\newcommand{\pt}{{\cal P}^\y}

% \newcommand{\ab}{\bar{{\cal A}}}
% \newcommand{\at}{\tilde{{\cal A}}}

\newcommand{\ab}{{\cal A}^\x}
\newcommand{\at}{{\cal A}^\y}




% \newtheoremstyle{exampstyle}
%   {\topsep} % Space above
%   {0} % Space below
%   {} % Body font
%   {} % Indent amount
%   {\bfseries} % Theorem head font
%   {.} % Punctuation after theorem head
%   {0pt} % Space after theorem head
%   {} % Theorem head spec (can be left empty, meaning `normal')

% \theoremstyle{exampstyle} 



\newcommand{\insertimage}[4]{ % scale, filename, caption, label
\begin{figure}[t]
\centering
\includegraphics[scale=#1, clip=true]{figures/#2}
\caption{#3}
\label{#4}
\end{figure}
}

\newcommand{\insertimageC}[5]{ % scale, filename, caption, label, location
\begin{figure}[#5]
\centering
\includegraphics[width=#1\linewidth, clip=true]{figures/#2}
%\vspace{-1.5em}
\caption{#3}
%\vspace{-0.5em}
\label{#4}
\end{figure}
}

%\setlength{\belowcaptionskip}{-10pt}
%\captionsetup[figure]{calcwidth = 0.95\linewidth,skip=10pt}
% \setlength{\abovecaptionskip}{-10pt}

\newcommand{\insertimageStar}[5]{ % scale, filename, caption, label, location
\begin{figure*}[#5]
\centering
\includegraphics[width=#1\linewidth, clip=true]{figures/#2}
\caption{#3}
%\vspace{-0.5em}
\label{#4}
\end{figure*}
}

\newtheorem{assumption}{\textbf{H}\hspace{-3pt}}
\Crefname{assumption}{\textbf{H}\hspace{-3pt}}{\textbf{H}\hspace{-3pt}}
\crefname{assumption}{\textbf{H}}{\textbf{H}}


\newcommand{\insertimageAsSubfig}[5]{ % scale, filename, caption, label,
\begin{figure}[#5]
\begin{center}
\subfigbottomskip =-4in
\subfigure{
\includegraphics[width=#1\columnwidth]{figures/#2}
\label{#4}
}
\end{center}
\subfigbottomskip =-4in
\caption{#3}
\label{#4}
\end{figure}
}


\newcommand{\floor}[1]{\left\lfloor #1 \right\rfloor}
\newcommand{\ceil}[1]{\left\lceil #1 \right\rceil}

\newcommand{\ps}[2]{\left\langle#1,#2 \right\rangle}
\newcommand{\coint}[1]{\left[#1\right)}
\newcommand{\ocint}[1]{\left(#1\right]}
\newcommand{\ooint}[1]{\left(#1\right)}
\newcommand{\ccint}[1]{\left[#1\right]}
\def\eg{e.g.}

%%% mathsf
\def\msi{\mathsf{I}}
\def\msa{\mathsf{A}}
\def\msd{\mathsf{D}}
\def\msk{\mathsf{K}}
\def\mss{\mathsf{S}}
\def\msn{\mathsf{N}}
\def\msat{\tilde{\mathsf{A}}}
\def\msb{\mathsf{B}} 
\def\msc{\mathsf{C}}
\def\mse{\mathsf{E}}
\def\msf{\mathsf{F}}
\def\mso{\mathsf{o}}
\def\msg{\mathsf{G}}
\def\msh{\mathsf{H}}
\def\msm{\mathsf{M}}
\def\msu{\mathsf{U}}
\def\msv{\mathsf{V}}
\def\msr{\mathsf{R}}
\newcommand{\msff}[2]{\mathsf{F}_{#1}^{#2}}
\def\msp{\mathsf{P}}
\def\msq{\mathsf{Q}}
\def\msx{\mathsf{X}}
\def\msy{\mathsf{Y}}



%% mathcal
\def\mca{\mathcal{A}}
\def\mcat{\tilde{\mathcal{A}}}
\def\mcab{\bar{\mathcal{A}}}
\def\mcbb{\mathcal{B}}  %%% \mcb est déjà pris
\newcommand{\mcb}[1]{\mathcal{B}(#1)}
\def\mcc{\mathcal{C}}
\def\mcy{\mathcal{Y}}
\def\mcx{\mathcal{X}}
\def\mce{\mathcal{E}}
\def\mcf{\mathcal{F}}
\def\mcg{\mathcal{G}}
\def\mch{\mathcal{H}}
\def\mcm{\mathcal{M}}
\def\mcu{\mathcal{U}}
\def\mcv{\mathcal{V}}
\def\mcr{\mathcal{R}}
\newcommand{\mcff}[2]{\mathcal{F}_{#1}^{#2}}
\def\mcfb{\bar{\mathcal{F}}}
\def\bmcf{\bar{\mathcal{F}}}
\def\mcft{\tilde{\mathcal{F}}}
\def\tmcf{\tilde{\mathcal{F}}}
\def\mcp{\mathcal{P}}
\def\mcq{\mathcal{Q}}

%% mathbb

\def\rset{\mathbb{R}}
\def\rsets{\mathbb{R}^*}
\def\cset{\mathbb{C}}
\def\zset{\mathbb{Z}}
\def\nset{\mathbb{N}}
\def\nsets{\mathbb{N}^*}
\def\qset{\mathbb{Q}}
\def\Rset{\mathbb{R}}
\def\Cset{\mathbb{C}}
\def\Zset{\mathbb{Z}}
\def\Nset{\mathbb{N}}
\def\Tset{\mathbb{T}}


%%%% mathrm 

\def\rmd{\mathrm{d}}
\def\mrd{\mathrm{d}}
\def\mrl{\mathrm{L}}
\def\mre{\mathrm{e}}
\def\rme{\mathrm{e}}
\def\rmn{\mathrm{n}}
\def\mrn{\mathrm{n}}
\def\mrc{\mathrm{C}}
\def\mrcc{\mathrm{c}}
\def\rmc{\mathrm{C}}
\def\rma{\mathrm{a}}
\def\mra{\mathrm{a}}

\newcommandx{\norm}[2][1=]{\ifthenelse{\equal{#1}{}}{\left\Vert #2 \right\Vert}{\left\Vert #2 \right\Vert^{#1}}}
\newcommandx{\normLigne}[2][1=]{\ifthenelse{\equal{#1}{}}{\Vert #2 \Vert}{\Vert #2\Vert^{#1}}}

\def\plusinfty{+\infty}
\def\ie{\textit{i.e.}}

\def\divop{\operatorname{div}}

\begin{document}

%%%%%%%%%%%%%%%%%%%%%%%%%%%%%%%%%%%%%%%%%%%%%%%%%%%%%%%%%%%%%%%%%%%%%%%%%%%%%%
%%% Here starts the poster
%%%---------------------------------------------------------------------------
%%% Format it to your taste with the options
%%%%%%%%%%%%%%%%%%%%%%%%%%%%%%%%%%%%%%%%%%%%%%%%%%%%%%%%%%%%%%%%%%%%%%%%%%%%%%
% Define some colors

%\definecolor{lightblue}{cmyk}{0.83,0.24,0,0.12}
\definecolor{lightblue}{rgb}{0.145,0.6666,1}
\definecolor{tpt}{RGB}{200,34,84}

% Draw a video
\newlength{\FSZ}
\newcommand{\drawvideo}[3]{% [0 0.25 0.5 0.75 1 1.25 1.5]
   \noindent\pgfmathsetlength{\FSZ}{\linewidth/#2}
   \begin{tikzpicture}[outer sep=0pt,inner sep=0pt,x=\FSZ,y=\FSZ]
   \draw[color=lightblue!50!black] (0,0) node[outer sep=0pt,inner sep=0pt,text width=\linewidth,minimum height=0] (video) {\noindent#3};
   \path [fill=lightblue!50!black,line width=0pt] 
     (video.north west) rectangle ([yshift=\FSZ] video.north east) 
    \foreach \x in {1,2,...,#2} {
      {[rounded corners=0.6] ($(video.north west)+(-0.7,0.8)+(\x,0)$) rectangle +(0.4,-0.6)}
    }
;
   \path [fill=lightblue!50!black,line width=0pt] 
     ([yshift=-1\FSZ] video.south west) rectangle (video.south east) 
    \foreach \x in {1,2,...,#2} {
      {[rounded corners=0.6] ($(video.south west)+(-0.7,-0.2)+(\x,0)$) rectangle +(0.4,-0.6)}
    }
;
   \foreach \x in {1,...,#1} {
     \draw[color=lightblue!50!black] ([xshift=\x\linewidth/#1] video.north west) -- ([xshift=\x\linewidth/#1] video.south west);
   }
   \foreach \x in {0,#1} {
     \draw[color=lightblue!50!black] ([xshift=\x\linewidth/#1,yshift=1\FSZ] video.north west) -- ([xshift=\x\linewidth/#1,yshift=-1\FSZ] video.south west);
   }
   \end{tikzpicture}
}

\hyphenation{resolution occlusions}
%%
\begin{poster}%
  % Poster Options
  {
  % Show grid to help with alignment
  grid=true,
  columns=3,
  % Column spacing
  colspacing=0.35em,
  % Color style
  bgColorOne=white,
  bgColorTwo=white,
  borderColor=purple,
  headerColorOne=black,
  headerColorTwo=purple,
  headerFontColor=white,
  boxColorOne=white,
  boxColorTwo=lightblue,
  % Format of textbox
  textborder=roundedleft,
  % Format of text header
  eyecatcher=true,
  headerborder=closed,
  headerheight=0.131\textheight,
%  textfont=\sc, An example of changing the text font
  headershape=roundedright,
  headershade=shadelr,
  headerfont=\Large\bf\textsc, %Sans Serif
  textfont={\setlength{\parindent}{1.5em}},
  boxshade=plain,
 % background=shadelr,
  background=plain,
  linewidth=2pt
  }
  % Eye Catcher
  { 
  % \begin{minipage}[h]{0.1\textwidth} \centering $\vcenter{\includegraphics[height=3em]{images/Logo-Inria.jpg}}$ \end{minipage}  
  \includegraphics[height=5em]{images/inria3.png} 
  \includegraphics[height=7em]{images/telecomparis_endossem_ipp_rvb_600pix.png}} 
  % Title
  {  \huge \bf\textsc{Sliced-Wasserstein Flows: Nonparametric Generative Modeling via Optimal Transport and Diffusions} \vspace{3pt} }
  % Authors
  {\large {Antoine Liutkus$^1$, \hspace{4pt} Umut \c Sim\c sekli$^2$, \hspace{4pt} Szymon Majewski$^3$, \hspace{4pt} Alain Durmus$^4$, \hspace{4pt} Fabian-Robert St\"{o}ter$^1$} \vspace{4pt}\\
   \normalsize{\textbf{1:} Inria, \textbf{2:} T\'{e}l\'{e}com Paris, Institut Polytechnique de Paris, \textbf{3:} Polish Academy of Sciences, \textbf{4:} CMLA, ENS Paris-Saclay} \vspace{3pt} \\
  % {\small Supported by the French National Research Agency (\textbf{ANR}) as a part of the \textbf{FBIMATRIX} project (ANR-16-CE23-0014)}
  }
  % University logo
  {% The makebox allows the title to flow into the logo, this is a hack because of the L shaped logo.
    %\includegraphics[height=7.5em]{images/logo_sb.pdf}
    %\includegraphics[height=3.0em]{images/bilgi.jpg}
    \includegraphics[height=3.2em]{images/pas.jpg} 
     \includegraphics[height=3.2em]{images/enslogo.jpg} 
  }

%%%%%%%%%%%%%%%%%%%%%%%%%%%%%%%%%%%%%%%%%%%%%%%%%%%%%%%%%%%%%%%%%%%%%%%%%%%%%%
%%% Now define the boxes that make up the poster
%%%---------------------------------------------------------------------------
%%% Each box has a name and can be placed absolutely or relatively.
%%% The only inconvenience is that you can only specify a relative position 
%%% towards an already declared box. So if you have a box attached to the 
%%% bottom, one to the top and a third one which should be in between, you 
%%% have to specify the top and bottom boxes before you specify the middle 
%%% box.
%%%%%%%%%%%%%%%%%%%%%%%%%%%%%%%%%%%%%%%%%%%%%%%%%%%%%%%%%%%%%%%%%%%%%%%%%%%%%%
    %
    % A coloured circle useful as a bullet with an adjustably strong filling
    \newcommand{\colouredcircle}{%
      \tikz{\useasboundingbox (-0.2em,-0.32em) rectangle(0.2em,0.32em); \draw[draw=black,fill=purple,line width=0.03em] (0,0) circle(0.18em);}}


 %%%%%%%%%%%%%%%%%%%%%%%%%%%%%%%%%%%%%%%%%%%%%%%%%%%%%%%%%%%%%%%%%%%%%%%%%%%%%%
  \headerbox{Motivation and Goal}{name=intro,column=0,row=0}{
%%%%%%%%%%%%%%%%%%%%%%%%%%%%%%%%%%%%%%%%%%%%%%%%%%%%%%%%%%%%%%%%%%%%%%%%%%%%%%

\begin{flushleft}
\noindent \colouredcircle $\>$  Generative model nedir, LMC, FPE connections
\end{flushleft}

}


% %%%%%%%%%%%%%%%%%%%%%%%%%%%%%%%%%%%%%%%%%%%%%%%%%%%%%%%%%%%%%%%%%%%%%%%%%%%%%%
%   \headerbox{Method of Analysis}{name=methods,column=0,row=0.475}{
% %%%%%%%%%%%%%%%%%%%%%%%%%%%%%%%%%%%%%%%%%%%%%%%%%%%%%%%%%%%%%%%%%%%%%%%%%%%%%%

  

% \begin{flushleft}
% \noindent \colouredcircle $\>$ 
% \end{flushleft}
% }
 


%%%%%%%%%%%%%%%%%%%%%%%%%%%%%%%%%%%%%%%%%%%%%%%%%%%%%%%%%%%%%%%%%%%%%%%%%%%%%%
  \headerbox{Assumptions \& Intermediate Results }{name=assumptions,column=1,row=0}{
%%%%%%%%%%%%%%%%%%%%%%%%%%%%%%%%%%%%%%%%%%%%%%%%%%%%%%%%%%%%%%%%%%%%%%%%%%%%%%
\begin{flushleft}



\end{flushleft}
}



 %%%%%%%%%%%%%%%%%%%%%%%%%%%%%%%%%%%%%%%%%%%%%%%%%%%%%%%%%%%%%%%%%%%%%%%%%%%%%%
  \headerbox{Main Result}{name=res,column=2,row=0,span=1}{
%%%%%%%%%%%%%%%%%%%%%%%%%%%%%%%%%%%%%%%%%%%%%%%%%%%%%%%%%%%%%%%%%%%%%%%%%%%%%%


}


 %%%%%%%%%%%%%%%%%%%%%%%%%%%%%%%%%%%%%%%%%%%%%%%%%%%%%%%%%%%%%%%%%%%%%%%%%%%%%%
  \headerbox{Additional Results}{name=extension,column=2,row=0.328,span=1}{
%%%%%%%%%%%%%%%%%%%%%%%%%%%%%%%%%%%%%%%%%%%%%%%%%%%%%%%%%%%%%%%%%%%%%%%%%%%%%%

\noindent \colouredcircle $\>$ 
}


 %%%%%%%%%%%%%%%%%%%%%%%%%%%%%%%%%%%%%%%%%%%%%%%%%%%%%%%%%%%%%%%%%%%%%%%%%%%%%%
  \headerbox{References}{name=ref,column=2,row=0.83}{
%%%%%%%%%%%%%%%%%%%%%%%%%%%%%%%%%%%%%%%%%%%%%%%%%%%%%%%%%%%%%%%%%%%%%%%%%%%%%%
% {
% \fontsize{6}{6}\selectfont
% \noindent [1] \c Sim\c sekli, U. "Fractional Langevin Monte Carlo: Exploring Levy Driven Stochastic Differential Equations for Markov Chain Monte Carlo." ICML 2017.\\
% \noindent [2] Raginsky, M., Rakhlin, A., Telgarsky, M. "Non-convex learning via Stochastic Gradient Langevin Dynamics: a nonasymptotic analysis." COLT 2017. \\
% \noindent [3] Simsekli U., Sagun L., Gurbuzbalaban, M. "A Tail-Index Analysis of Stochastic Gradient Noise in Deep Neural Networks." ICML 2019.
% }
}


\end{poster}

\end{document}

